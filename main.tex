\documentclass[system, master]{systemM}
%\documentclass[oneside]{suribt}% 本文が * ページ以下のときに (掲示に注意)
\bibliographystyle{junsrt} 

\usepackage[dvipdfmx]{graphicx,color}
\usepackage{url}
\usepackage{amsmath,amssymb,amsthm}
\usepackage{mathtools}
\usepackage[normalem]{ulem}

\usepackage{subfiles} % Best loaded last in the preamble

\title{タイトル}
%\titlewidth{}% タイトル幅 (指定するときは単位つきで)
\author{かせがお}
\eauthor{Kasegao}% Copyright 表示で使われる
\studentid{学生証番号}
\supervisor{指導教員名 役職}% 1 つ引数をとる (役職まで含めて書く)
%\supervisor{指導教員名 役職 \and 指導教員名 役職}% 複数教員の場合,\and でつなげる
\handin{2016}{1}% 提出月. 2 つ (年, 月) 引数をとる
%\keywords{キーワード1, キーワード2} % 概要の下に表示される

\begin{document}
\maketitle%%%%%%%%%%%%%%%%%%% タイトル %%%%

\frontmatter% ここから前文

\etitle{Title in English}

\begin{eabstract}%%%%%%%%%%%%% 概要 %%%%%%%%
 300 words abstract in English should be written here. 
\end{eabstract}

\begin{abstract}%%%%%%%%%%%%% 概要 %%%%%%%%
 ここに概要を書く.
\end{abstract}


%%%%%%%%%%%%% 目次 %%%%%%%%
%\tableofcontents
{\makeatletter
\let\ps@jpl@in\ps@empty
\makeatother
\pagestyle{empty}
\tableofcontents
\clearpage}

\mainmatter% ここから本文 %%% 本文 %%%%%%%%
\subfile{chapters/symbols}

\backmatter% ここから後付
\chapter{謝辞}%%%%%%%%%%%%%%% 謝辞 %%%%%%%

\bibliography{references}

\appendix% ここから付録 %%%%% 付録 %%%%%%%
\chapter{}
\end{document}
